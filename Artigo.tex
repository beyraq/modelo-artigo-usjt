%%%%%%%%%%%%%%%%%%%%%%%%%%%%%%%%%%%%%%%%%%%%%%%%%%%%%%%%%%%%%%%%%%%%%%%%%%%%%%%%%%%
% Thiago Pelizoni
%%%%%%%%%%%%%%%%%%%%%%%%%%%%%%%%%%%%%%%%%%%%%%%%%%%%%%%%%%%%%%%%%%%%%%%%%%%%%%%%%%%
% Modelo de Artigo utilizado em minha pós-graduação em Engenharia de Software na Universidade São Judas Tadeu
%%%%%%%%%%%%%%%%%%%%%%%%%%%%%%%%%%%%%%%%%%%%%%%%%%%%%%%%%%%%%%%%%%%%%%%%%%%%%%%%%%%

\documentclass[12pt, a4paper]{article}				

\usepackage[utf8]{inputenc}
\usepackage[brazil]{babel}	
\usepackage{graphicx,color}
\usepackage[left=2.0cm,top=2.5cm,right=2.0cm,bottom=2.0cm]{geometry}
\usepackage{enumerate}
\usepackage{epigraph}
\usepackage{abntcite}

% Título do trabalho
% O título deve ser curto, claro e conciso. Um bom título descreve o conteúdo do artigo
\title{Titulo do Artigo}

% Autor
\author{Seu nome}

\date{2013}


%%%%%%%%%%%%%%%%%%%%%%%%%%%%%%%%%%%%%%%%%%%%%%%%%%%%%%%%%%%%%%%%%%%%%%%%%%%%%%%%%%%

\begin{document}

% Faz a criação do título do documento
\maketitle

%%%%%%%%%%%%%%%%%%%%%%%%%%%%%%%%%%%%%%%%%%%%%%%%%%%%%%%%%%%%%%%%%%%%%%%%%%%%%%%%%%%
% Resumo: Resume os principais aspectos do artigo
%%%%%%%%%%%%%%%%%%%%%%%%%%%%%%%%%%%%%%%%%%%%%%%%%%%%%%%%%%%%%%%%%%%%%%%%%%%%%%%%%%%
\begin{abstract}
Um bom resumo deve conter: Contextualização, Lacuna, Proposta, Metodologia, Resultados e Conclusão.

\begin{itemize}
	\item Usar o tempo Passado sempre que possível;
	\item Usar voz ativa preferencialmente;
	\item Ser conciso e utilizar frases completas
\end{itemize}

\end{abstract}


%%%%%%%%%%%%%%%%%%%%%%%%%%%%%%%%%%%%%%%%%%%%%%%%%%%%%%%%%%%%%%%%%%%%%%%%%%%%%%%%%%%
% Introdução: Descreve porquê o trabalho foi feito
%%%%%%%%%%%%%%%%%%%%%%%%%%%%%%%%%%%%%%%%%%%%%%%%%%%%%%%%%%%%%%%%%%%%%%%%%%%%%%%%%%%
\section{Introdução}
\begin{enumerate}
	\item \textit{Contextualização}: Apresentar o campo de pesquisa e mostrar a importância da área principal, os prazos e os processos familiares;
	\item \textit{Declare a Lacuna}: Perguntas abertas, restrições e limitações;
	\item \textit{Mostrar o Estado da Arte}: Evidenciar recentes pesquisas e descobertas;
	\item \textit{Afirmar a importância de seu estudo}: Implicações evidenciando e/ou aplicações;
	\item \textit{Proposta}: Descrever a proposta do artigo com frases como:
	\begin{itemize}
		\item No trabalho aqui relatado ...
		\item Neste trabalho, as características detalhadas do ...
	\end{itemize}
\end{enumerate}

As informações no texto devem fluir de um geral para algo específico, chegando ao fim. Citar autores afim de indicar onde suas ideias vieram.

Eis um exemplo de citação\cite[p. 215]{pressman}

%%%%%%%%%%%%%%%%%%%%%%%%%%%%%%%%%%%%%%%%%%%%%%%%%%%%%%%%%%%%%%%%%%%%%%%%%%%%%%%%%%%
% Metodologia ou Materiais e Métodos: Descreve como o trabalho foi feito
%%%%%%%%%%%%%%%%%%%%%%%%%%%%%%%%%%%%%%%%%%%%%%%%%%%%%%%%%%%%%%%%%%%%%%%%%%%%%%%%%%%
\section{Metodologia}
Descrever como o trabalho foi feito.

%%%%%%%%%%%%%%%%%%%%%%%%%%%%%%%%%%%%%%%%%%%%%%%%%%%%%%%%%%%%%%%%%%%%%%%%%%%%%%%%%%%
% Resultados: O que foi observado?
%%%%%%%%%%%%%%%%%%%%%%%%%%%%%%%%%%%%%%%%%%%%%%%%%%%%%%%%%%%%%%%%%%%%%%%%%%%%%%%%%%%
\section{Resultados}
Resultados, o que foi observado.

%%%%%%%%%%%%%%%%%%%%%%%%%%%%%%%%%%%%%%%%%%%%%%%%%%%%%%%%%%%%%%%%%%%%%%%%%%%%%%%%%%%
% 4. Considerações finais - o que se pode concluir dos resultados?
%%%%%%%%%%%%%%%%%%%%%%%%%%%%%%%%%%%%%%%%%%%%%%%%%%%%%%%%%%%%%%%%%%%%%%%%%%%%%%%%%%%
\section{Conclusão}
Considerações finais, o que se pode concluir dos resultados?

% Faz a criação das referências bibliográficas
\bibliography{Bibliografia}

\end{document}
